\documentclass[10pt,a4paper]{article}

\input{AEDmacros}
\usepackage{caratula} % Version modificada para usar las macros de algo1 de ~> https://github.com/bcardiff/dc-tex


\titulo{Trabajo práctico 1}
\subtitulo{Especificación y WP}

\fecha{\today}

\materia{Algoritmos y estructuras de datos}
\grupo{}

\integrante{Llop, Tobias}{871/22}{tobiasllop@gmail.com}
\integrante{Pasquet, Felipe Luc}{002/01}{email2@dominio.com}
\integrante{Catarasso, Sofia}{003/01}{email3@dominio.com}
\integrante{Dalbene, Héctor Anselmo}{004/01}{email4@dominio.com}
% Pongan cuantos integrantes quieran

% Declaramos donde van a estar las figuras
% No es obligatorio, pero suele ser comodo
\graphicspath{{../static/}}

\begin{document}

\maketitle

\section{Especificación}
\subsection{Ejercicio 1: \( \textmd{hayBallotage} \)}

\begin{proc}{hayBallotage}{\In escrutinio : \TLista{\ent}}{\bool}
	%    \modifica{parametro1, parametro2,..}
	\requiere{|escrutinio| > 0}
	\asegura{res = \True \iff ( \neg \existe[unalinea]{i}{\ent}{0 \leq i < |escrutinio| \yLuego porcentaje(escrutinio, i) > 45}\\
	         \yLuego \paraTodo[unalinea]{i}{\ent}{0 \leq i < |escrutinio| \yLuego 40 < porcentaje(escrutinio, i) \leq 45} \implicaLuego\\ 
			 \existe[unalinea]{j}{\ent}{0 \leq j < |escrutinio| \yLuego i \neq j \yLuego porcentaje(escrutinio, i) - 10 < porcentaje(escrutinio, j)})}
	\aux{porcentaje}{s: \TLista{\ent}, i : \ent}{\float}{\frac{s[i] \: * \:100}{\sum_{k= 0}^{|s|-1}s[k]}}
\end{proc}

\subsection{Ejercicio 2: \textmd{HayFraude}}


\begin{proc}{hayFraude}{\In escrutinioPres: \TLista{\ent}, \In escrutinioSen: \TLista{\ent}, \In escrutinioDip: \TLista{\ent}}{\bool}	%    \modifica{parametro1, parametro2,..}
	\requiere{|escrutinioPres| \geq 1, |escrutinioSen| \geq 1, |escrutinioDip| \geq 1}
	\asegura{res = \False \iff \\
	\sum_{i=0}^{|escrutinioPres| -1}escrutinioPres[i] = \sum_{j=0}^{|escrutinioSen| -1}escrutinioSen[j] = \sum_{k=0}^{|escrutinioDip| -1}escrutinioDip[k] }
\end{proc}

\subsection{Ejercicio 3: \textmd{obtenerSenadoresEnProvincia}}

\begin{proc}{obtenerSenadoresEnProvincia}{\In escrutinio: \TLista{\ent}}{\ent x\ent}	%    \modifica{parametro1, parametro2,..}
	\requiere{|escrutinio| \geq 3 \yLuego noHayEmpate(escrutinio)}
	\asegura{res_0 = esMaximo(escrutinio) \,\land\,\,  res_1 = esMaximo(quitar(escrutinio, escrutinio[esMaximo(escrutinio)]))}
\end{proc}
\:

\pred{noHayEmpate}{s: \TLista{\ent}}{\paraTodo[unalinea]{i}{\ent}{0 \leq i < |escrutinio| -1 \land \paraTodo[unalinea]{j}{\ent}{0 \leq j < |escrutinio| -1 \land i \neq j \land s[i] \neq s[j]}}}

\:

\aux{esMaximo}{\In escrutinio: \TLista{\ent}}{\ent}{\existe[unalinea]{i}{\ent}{0 \leq i < |escrutinio| - 1 \land \\ 
\setlength{\parindent}{24mm}\indent \paraTodo[unalinea]{j}{\ent}{0 \leq j < |escrutinio| -1 \land escrutinio[i] > escrutino[j] \yLuego res = i}}}

\:

\aux{quitar}{\In escrutinio: \TLista{\ent}, pos: \ent}{\TLista{\ent}}{addFirst(escrutinio[|escrutinio|-1], res) \yLuego \\
\setlength{\parindent}{20mm}\indent \paraTodo[unalinea]{j}{\ent}{0 \leq j < |escrutinio|-1 \land j \neq pos \land addFirst(escrutinio[j], res)}}

\subsection{Ejercicio 4: \textmd{calcularDHontEnProvincia}}


\begin{proc}{calcularDHontEnProvincia}{\In cantBancas: \ent,  \In escrutinio: \TLista{\ent}}{\TLista{\ent}}	%    \modifica{parametro1, parametro2,..}
	\requiere{|escrutinio| > 1 \land cantBancas > 0 \yLuego noHayEmpate(escrutinio) }
	\asegura{\paraTodo[unalinea]{i, j}{\ent}{0 \leq j < CantBancas \land 0 \leq j < |escrutinio| - 1 \implica res[i][j] =  \frac{escrutinio[i]}{j + 1}}}
\end{proc}


\subsection{Ejercicio 5: \textmd{calcularDHontEnProvincia}}
	\begin{proc}{obtenerDiputadosEnProvincia}{\In cantBancas: \ent, \In escrutinio: \ent, \In dHont \matriz{\ent}}{\TLista{\ent}}{
		\requiere{ cantBancas > 0 \land |escrutinio| > 1 \land |dHont| > 0}
		\asegura{|res| = |dHont| \land |res| = |escrutinio| - 1 \land \\
		\setlength{\parindent}{20mm} \paraTodo[unalinea]{i}{\ent}{0 \leq i < |res| \implicaLuego res[i] = \sum_{j = 1}^{CantBancas + 1} if \, \, sumaBanca(escrutinio[i]/j , dHont, cantBancas) \, \, then \, \, 1 \, \, else \, \,0 }		}
	}
	\end{proc}
\pred{sumaBanca}{\In cociente: \ent, \In dHont: \matriz{\ent}, \In CantBancas: \ent}{
	cantBancas > 0 \yLuego |dHont[0]| > 0 \implicaLuego (\sum_{i= 0}^{|dHont| - 1} \sum_{j=0}^{|dHont[0]| - 1}if \, \, cociente \leq dHont[i][j] \, \, then \, \, 1 \, \, else \, \,0) \leq CantBancas
}

	\subsection{Ejercicio 6: \textmd{validarListasEnProvincia}}
	\begin{proc}{validarListasEnProvincia}{\In cantBancas: \ent, \In listas: \matriz{dni: \ent\times genero: \ent}}{\bool}
		\requiere{cantBancas > 0 \land |listas| > 0}
		\asegura{\paraTodo[unalinea]{i}{\ent}{0\leq i < |listas| \yLuego |listas[i]| = CantBancas \land cumpleAlternancia(listas[i])} \iff res = \True}
	\end{proc}
	
	\pred{cumpleAlternancia}{\In lista: \TLista{dni: \ent\times genero: \ent}}{ \paraTodo[unalinea]{i}{\ent}{0 \leq i < |lista| - 2 \implicaLuego lista_1[i] \neq lista_1[i+1]}}

\section{Implementaciones y demostraciones de correctitud}
\subsection{Ejercicio 1 - Implementación}
\begin{minipage}[t]{\textwidth}
	hayBallotage \(\equiv\) \begin{lstlisting}
		i:= 0
		suma := 0
		while (i < escrutinio.size()) do
			suma := suma + escrutinio[i]
			i:= i + 1
		endwhile
		j := 0
		res := true
		k := 0
		porc := 100 / suma
		while (j < escrutinio.size() - 1) do 
			if  (escrutinio[j] * porc) > 45 then
				res := false
			endif
			if (escrutinio[j] * porc) <= 45 and (escrutinio[j] * porc) > 40 then
				while (k < escrutinio.size() -1) do
					if (escrutunio[j] * porc) - 10 > (escrutinio[k] * porc) then
						res := false
					endif
					k := k +1
				endwhile
			endif
			j += j + 1
		endwhile
		return res
	\end{lstlisting}
\end{minipage}		


\subsection{Ejercicio 2 - Implementación y WP}
\begin{minipage}[t]{\textwidth}
	hayFraude \(\equiv\) \begin{lstlisting}
	p := 0
	s := 0
	d := 0
	votosPresidente := 0
	votosSenador := 0
	votosDiputado := 0
	while p < escrutinio_presidencial.size() do
		votosPresidente = votosPresidente + escrutinio_presidencial[p]
		p = p+1
	endwhile	
	
	while s < escrutinio_senadores.size() do
		votosSenador = votosSenador + escrurinio_senadores[s]
		s = s+1
	endwhile	
	
	while d < escrutinio_diputados.size() do
		votosDiputado = votosDiputado + escrutinio_diputados[d]
		d = d+1
	endwhile
		
	if (votosDiputado = votosSenador) and (votosDiputado = votosPresidente) then
		res := True
	else
		res := False
	return res
	\end{lstlisting}
\end{minipage}		
\subsection{Ejercicio 3 - Implementación y WP}
None
\subsection{Ejercicio 4 - Implementación}
None
\subsection{Ejercicio 5 - Implementación}
None
\subsection{Ejercicio 6 - Implementación}
\begin{minipage}[t]{\textwidth}
	validarListasEnProvincia \(\equiv\) \begin{lstlisting}
	res := true
	while i < listas.size() do
		if cantBancas != listas[i].size() then
			res:= false
		else
			skip
		while j < listas[i].size() - 2 do
			if lista[i][j] = lista[i][j+1] then
				res := false
			else
				skip
			j = j + 1
		endwhile
		i = i + 1
	endwhile
	return res
	\end{lstlisting}
\end{minipage}	


\end{document}



\subsection{Ejercicio 6 - Implementación}
\begin{minipage}[t]{\textwidth}
	validarListasEnProvincia \(\equiv\) \begin{lstlisting}
	res := true
	while i < listas.size() do
		if cantBancas != listas[i].size() then
			res:= false
		else
			skip
		while j < listas[i].size() - 2 do
			if lista[i][j] = lista[i][j+1] then
				res := false
			else
				skip
			j = j + 1
		endwhile
		i = i + 1
	endwhile
	\end{lstlisting}
\end{minipage}	